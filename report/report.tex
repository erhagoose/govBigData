\documentclass[a4paper,12pt]{article}
\usepackage[notoc,noabs]{HaotianReport}

\title{基于文本数据挖掘的官媒主题效力分析}
\author{刘昊天}
\authorinfo{lht18@mails.tsinghua.edu.cn}
\runninghead{清华大学《政务大数据》2019秋季学期}
\studytime{2019年12月}

\begin{document}
    \maketitle
    \section{研究问题}
    随着智能手机的不断普及,移动社交软件已经成为人们生活的重要组成部分,而微信作为目前腾讯旗下赶超 QQ 的社交软件,更是在前几年迎来爆发式增长.根据腾讯控股 [00700]2018年年报,截止至 2018 年 12 月 31,微信合并月活跃用户数达 10.98 亿,同比增长 11.0\%;同时, 2018 年腾讯的网络广告业务收入达 581 亿元,同比增长 44\%.在微信及腾讯广告业务的发展过程中,微信公众号功不可没,其已经成为了国内最大的内容提供平台之一.个人或企业用户可以在微信公众号上发布文章,通过微信的订阅服务定向推送给已订阅的用户,无论是宣传还是营收都有巨大的空间;已订阅用户则可以消费微信公众号的推送内容.由此可见,微信公众号已经并将在很长一段时间内继续扮演中国人生活中的重要角色.

    与此对应的,作为个人用户,我们对某一特定的微信公众号并没有方便的认知渠道.当我们看到某一个公众号,我们只能通过阅读简介或查看朋友关注数、历史文章数等方式对一个公众号进行直观的了解,并可能阅览有限的几篇历史文章进行进一步了解.这种程度的认知,并不足以为我们判断其内容质量、权威性、主题、同自身兴趣契合程度等提供充分的指导.因此,如果能对微信公众号的数据进行分析与可视化,将该公众号的内容简洁、直观、充分地呈现给用户,则将在精准内容选择方面提供巨大的帮助.

    2019 年 3 月 31 日,咪蒙发布朋友圈调侃“第二次开垮公司”,宣告其旗下的北京〸月初五影视公司正式解散;而就在 2018 年底,咪蒙公开数据称其公众号"1400 万粉丝,单篇最高阅读量为 1470 万".朋友圈中,一度出现了" 含咪率"(个人朋友中关注咪蒙公众号的比例)这样调侃的词汇.可见公众号对舆论影响之重大,更凸显了对公众号分析的必要性.
    \section{研究方法}

    \section{数据来源及变量测量}

    \section{研究发现及结果解释}

    \label{applastpage}
    % \newpage
    % \bibliography{report}
    % \bibliographystyle{unsrt}
\iffalse
\begin{itemize}[noitemsep,topsep=0pt]
%no white space
\end{itemize}
\begin{enumerate}[label=\Roman{*}.,noitemsep,topsep=0pt]
%use upper case roman
\end{enumerate}
\begin{multicols}{2}
%two columns
\end{multicols}
\begin{figure}
  \centering
  \includegraphics[width=0.9\linewidth]{}
  \caption{}
  \label{fig:}
\end{figure}
\fi
\end{document}
